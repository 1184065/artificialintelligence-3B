\chapter{Fungsi dan Kelas}
\section{Pemahanan Teori}
\begin{enumerate}
\item Fungsi adalah satu blok program yang terdiri dari nama fungsi, input variabel dan variabel kembalian.  Nama fungsi diawali dengandefdan setelahnya tanda titik dua. inputan fungsi adalah untuk menerima baris input dari user dan mengembalikannya dalam bentuk string. Fungsi yang tidak mengembalikan nilai biasanya disebut dengan prosedur.
\begin{lstlisting}[caption=Contoh penggunaan fungsi,label={lst:Syntak fungsi}]
def npm1():
    print("***   ***   *******   ******  ****** ******  ***")
    print("***   ***   **   **  **   **  **  **     **  ***")
    print("***   ***    ****   ********  **  ** ******  ***")
    print("***   ***   **   **       **  **  ** **      ***")
    print("***   ***   *******       **  ****** ******  ***")
    
npm1()
\end{lstlisting}
\item Paket merupakan sebuah modul yang berisi kode-kode python dan isi paket ini bisa kita impor ke dalam program
\begin{lstlisting}
 from kardus_minum import botol
\end{lstlisting}
\item
\begin{enumerate}
\item Kelas adalah struktur data yang digunakan untuk mendefinisikan objek serta menyimpan data bersama nilai-nilai dan perilaku (behavior), Kelas juga merupakan suatu entitas yang merupakan bentuk program dari suatu abstraksi untuk permasalahan dunia nyata, dan instans dari class merupakan realisasi dari beberapa objek. Jika dianalogikan, kelas itu merupakan blueprint ( cetak biru ) dari sebuah objek (instans).
\begin{lstlisting}[caption=contoh class, label={lst:Syntak kelas}]
  class <nama_kelas> :
   <statemen>
   <statemen>
\end{lstlisting}
\item Objek memiliki variabel dan kode yang saling terhubung. objek di buat dengan class.

\item Atribut merupakan data atau bisa juga berupa fungsi-fungsi yang dimiliki oleh kelas tersebut. Atribut diakses melalui notasi bertitik. Atribut-atribut kelas terikat hanya untuk kelas-kelas dimana atribut tersebut didefinisikan. 
\begin{lstlisting}[caption= contoh atribut, label={lst:Syntak atribut}]
  >>> class X:
   ...     bil = 100
   ...
   >>> print X.bil
   100
   >>> X.bil = X.bil + 10
   >>> print X.bil
   110
\end{lstlisting}
\item  Method merupakan fungsi yang melekat pada sebuah objek atau instan kelas. Contoh berikut menunjukkan penggunaan method dalam kelas.
\begin{lstlisting}[caption= contoh method, label={lst:Syntak method}]
#Badan Class
  class TestMethod:
   def perkalian(self,a,b):
    c = a * b
    return c
  
  #program Utama
  objek = TestMethod()   #instansiasi objek
  print(objek.perkalian(50,2))
\end{lstlisting}
\end{enumerate}
\item Penggunaan Import pada kelas library hampir sama, tetapi untuk perbedaanya menggunakan penambahan variabel yang menjadi objek dari kelas.
\begin{lstlisting}
Import kotak

start=keluar.sesuatu()
hasil=start.ambil()
\end{lstlisting} 
\item program memanggil sebuah package langkah awal menambahkan syntak code penambahan.
\begin{lstlisting}
from scr import plus
\end{lstlisting}
\item untuk mengakses sebuah library dalam sebuah folder lain, perlu menulis kan nama folder kemudian mengimport nama librarynya:
\begin{lstlisting}
from kim kuproy import npm3
\end{lstlisting}
\item sama seperti mengimport library yaitu dengan menuliskan nama folder kemudian mengimportkan nama class tersebut : 
\begin{lstlisting}
from tugas import tugas1
\end{lstlisting}
\end{enumerate}
\section{Ketrampilan Pemrograman}
\begin{enumerate}
\item soal no 1
\begin{lstlisting}
def npm1():
    print("***   ***   *******   ******  ****** ******  ***")
    print("***   ***   **   **  **   **  **  **     **  ***")
    print("***   ***    ****   ********  **  ** ******  ***")
    print("***   ***   **   **       **  **  ** **      ***")
    print("***   ***   *******       **  ****** ******  ***")
npm1()
\end{lstlisting}
\item soal no 2
\begin{lstlisting}
def npm2(npm):
    npm=int(npm)
    TwoLastDigit=abs(npm)%100
    for i in range(TwoLastDigit):
       print("Halo, ", npm, " apa kabar ?")
\end{lstlisting}
\item soal no 3
\begin{lstlisting}
def npm3(npm):
    for i in range(int(str(npm)[4])+int(str(npm)[5])+int(str(npm)[6])):
        print("Halo, "+str(npm)[4]+str(npm)[5]+str(npm)[6]+" apa kabar ?")

i=0
npm=input("Masukan NPM : ")
while i<1:
    if len(npm) < 7:
        print("NPM Kurang dari 7 digit")
        npm=input("Masukan NPM : ")
    elif len(npm) > 7:
        print("NPM lebih dari 7 digit")
        npm=input("Masukan NPM : ")
    else:
        i=1
npm3(npm)
\end{lstlisting}
\item soal no 4
\begin{lstlisting}
def npm4(npm):
    key=npm%1000
    str_key=str(key)
    print("Halo, "+str_key[0]+" apa kabar ?")

i=0
npm=input("Masukan NPM : ")
while i<1:
    if len(npm) < 7:
        print("NPM Kurang dari 7 digit")
        npm=input("Masukan NPM : ")
    elif len(npm) > 7:
        print("NPM lebih dari 7 digit")
        npm=input("Masukan NPM : ")
    else:
        i=1
npm4(npm)
\end{lstlisting}
\item soal no 5
\begin{lstlisting}
def npm5(npm):
    a=npm[0]
    b=npm[1]
    c=npm[2]
    d=npm[3]
    e=npm[4]
    f=npm[5]
    g=npm[6]

    for x in a,b,c,d,e,f,g:
        print(x)

i=0
npm=input("Masukan NPM : ")
while i<1:
    if len(npm) < 7:
        print("NPM Kurang dari 7 digit")
        npm=input("Masukan NPM : ")
    elif len(npm) > 7:
        print("NPM lebih dari 7 digit")
        npm=input("Masukan NPM : ")
    else:
        i=1
npm5(npm)
\end{lstlisting}
\item soal 6
\begin{lstlisting}
def npm6(npm):

    a=npm[0]
    b=npm[1]
    c=npm[2]
    d=npm[3]
    e=npm[4]
    f=npm[5]
    g=npm[6]
    y=0

    for x in a,b,c,d,e,f,g:
        y+=int(x)
    print(y)

i=0
npm=input("Masukan NPM : ")
while i<1:
    if len(npm) < 7:
        print("NPM Kurang dari 7 digit")
        npm=input("Masukan NPM : ")
    elif len(npm) > 7:
        print("NPM lebih dari 7 digit")
        npm=input("Masukan NPM : ")
    else:
        i=1
npm6(npm)
\end{lstlisting}
\item soal 7
\begin{lstlisting}
def npm7(npm):

    a=npm[0]
    b=npm[1]
    c=npm[2]
    d=npm[3]
    e=npm[4]
    f=npm[5]
    g=npm[6]
    conv=1

    for x in a,b,c,d,e,f,g:
        conv*=int(x)
    print(conv)

i=0
npm=input("Masukan NPM : ")
while i<1:
    if len(npm) < 7:
        print("NPM Kurang dari 7 digit")
        npm=input("Masukan NPM : ")
    elif len(npm) > 7:
        print("NPM lebih dari 7 digit")
        npm=input("Masukan NPM : ")
    else:
        i=1
npm7(npm)
\end{lstlisting}
\item soal 8
\begin{lstlisting}
def npm8(npm):
    a=npm[0]
    b=npm[1]
    c=npm[2]
    d=npm[3]
    e=npm[4]
    f=npm[5]
    g=npm[6]
    for x in a,b,c,d,e,f,g:
        if int(x)%2==0:
            if int(x)==0:
                x=""
            print(x,end ="")


i=0
npm=input("Masukan NPM : ")
while i<1:
    if len(npm) < 7:
        print("NPM Kurang dari 7 digit")
        npm=input("Masukan NPM : ")
    elif len(npm) > 7:
        print("NPM lebih dari 7 digit")
        npm=input("Masukan NPM : ")
    else:
        i=1
npm8(npm)
\end{lstlisting}
\item soal no 9
\begin{lstlisting}
def npm9(npm):
    a=npm[0]
    b=npm[1]
    c=npm[2]
    d=npm[3]
    e=npm[4]
    f=npm[5]
    g=npm[6]
    for x in a,b,c,d,e,f,g:
    
        if int(x)%2==1:
            print(x,end ="")

i=0
npm=input("Masukan NPM : ")
while i<1:
    if len(npm) < 7:
        print("NPM Kurang dari 7 digit")
        npm=input("Masukan NPM : ")
    elif len(npm) > 7:
        print("NPM lebih dari 7 digit")
        npm=input("Masukan NPM : ")
    else:
        i=1
npm9(npm)
\end{lstlisting}
\item soal no 10
\begin{lstlisting}
def npm10(npm):
    a=npm[0]
    b=npm[1]
    c=npm[2]
    d=npm[3]
    e=npm[4]
    f=npm[5]
    g=npm[6]
    for x in a,b,c,d,e,f,g:    
        if int(x) > 1:
            for i in range(2,int(x)):
                if (int(x) % i) == 0:
                    break
            else:
                print(int(x),end =""),

i=0
npm=input("Masukan NPM : ")
while i<1:
    if len(npm) < 7:
        print("NPM Kurang dari 7 digit")
        npm=input("Masukan NPM : ")
    elif len(npm) > 7:
        print("NPM lebih dari 7 digit")
        npm=input("Masukan NPM : ")
    else:
        i=1
npm10(npm)
\end{lstlisting}
\item soal no 11
\begin{lstlisting}
def npm1():
    print("***   ***   *******   ******  ****** ******  ***")
    print("***   ***   **   **  **   **  **  **     **  ***")
    print("***   ***    ****   ********  **  ** ******  ***")
    print("***   ***   **   **       **  **  ** **      ***")
    print("***   ***   *******       **  ****** ******  ***")
def npm2(npm):
    npm=int(npm)
    TwoLastDigit=abs(npm)%100
    for i in range(TwoLastDigit):
       print("Halo, ", npm, " apa kabar ?")
def npm3(npm):
    for i in range(int(str(npm)[4])+int(str(npm)[5])+int(str(npm)[6])):
        print("Halo, "+str(npm)[4]+str(npm)[5]+str(npm)[6]+" apa kabar ?")
    return None
def npm4(npm):
    key=npm%1000
    str_key=str(key)
    print("Halo, "+str_key[0]+" apa kabar ?")
def npm5(npm):
    a=npm[0]
    b=npm[1]
    c=npm[2]
    d=npm[3]
    e=npm[4]
    f=npm[5]
    g=npm[6]

    for x in a,b,c,d,e,f,g:
        print(x)

def npm6(npm):

    a=npm[0]
    b=npm[1]
    c=npm[2]
    d=npm[3]
    e=npm[4]
    f=npm[5]
    g=npm[6]
    y=0

    for x in a,b,c,d,e,f,g:
        y+=int(x)
    print(y)
def npm7(npm):

    a=npm[0]
    b=npm[1]
    c=npm[2]
    d=npm[3]
    e=npm[4]
    f=npm[5]
    g=npm[6]
    conv=1

    for x in a,b,c,d,e,f,g:
        conv*=int(x)
    print(conv)
def npm8(npm):
    a=npm[0]
    b=npm[1]
    c=npm[2]
    d=npm[3]
    e=npm[4]
    f=npm[5]
    g=npm[6]
    for x in a,b,c,d,e,f,g:
        if int(x)%2==0:
            if int(x)==0:
                x=""
            print(x,end ="")
def npm9(npm):
    a=npm[0]
    b=npm[1]
    c=npm[2]
    d=npm[3]
    e=npm[4]
    f=npm[5]
    g=npm[6]
    for x in a,b,c,d,e,f,g:
    
        if int(x)%2==1:
            print(x,end ="")
def npm10(npm):
    a=npm[0]
    b=npm[1]
    c=npm[2]
    d=npm[3]
    e=npm[4]
    f=npm[5]
    g=npm[6]
    for x in a,b,c,d,e,f,g:    
        if int(x) > 1:
            for i in range(2,int(x)):
                if (int(x) % i) == 0:
                    break
            else:
                print(int(x),end =""),
\end{lstlisting}
Maka hasil untuk outputnya adalah : 
\begin{verbatim}
Masukan NPM kalian : 1184021
***   ***   *******   ******  ****** ******  ***
***   ***   **   **  **   **  **  **     **  ***
***   ***    ****   ********  **  ** ******  ***
***   ***   **   **       **  **  ** **      ***
***   ***   *******       **  ****** ******  ***

Halo, 021 apa kabar ?
Halo, 021 apa kabar ?
Halo, 021 apa kabar ?
3
\end{verbatim}
\item soal no 12
\begin{lstlisting}
import lib3
class Kelas3ngitung:
    def __init__(self,npm):
        self.npm = npm
    def npm1(self):
        return lib3.npm1()
    def npm2(self):
        return lib3.npm2(self.npm)
    def npm3(self):
        return lib3.npm3(self.npm)
    def npm4(self):
        return lib3.npm4(self.npm)
    def npm5(self):
        return lib3.npm5(self.npm)
    def npm6(self):
        return lib3.npm6(self.npm)
    def npm7(self):
        return lib3.npm7(self.npm)
    def npm8(self):
        return lib3.npm8(self.npm)
    def npm9(self):
        return lib3.npm9(self.npm)
    def npm10(self):
        return lib3.npm10(self.npm)
\end{lstlisting}
Maka main.py nya adalah :
\begin{lstlisting}
import kelas3lib
import lib3
from kalkulator import penambahan
   
npm=input("Masukan NPM kalian : ")
i=0
while i<1:
    if len(npm) < 7:
        print("NPM Kurang dari 7 digit")
        npm=input("Masukan NPM kalian : ")
    elif len(npm) > 7:
        print("NPM lebih dari 7 digit")
        npm=input("Masukan NPM kalian : ")
    else:
        i=1

#Contoh pemanggilan fungsi pada class
cobakelas=kelas3lib.Kelas3ngitung(npm) 
hasilkelas=cobakelas.npm1()


print("")

#Contoh pemanggilan fungsi pada library
lib3.npm3(npm)
print(penambahan.tambah(1,2))

\end{lstlisting}
\end{enumerate}


\section{Ketrampilan Penanganan Error}
\begin{lstlisting}

#soal 1
def npm1():
    print("***   ***   *******   ******  ****** ******  ***")
    print("***   ***   **   **  **   **  **  **     **  ***")
    print("***   ***    ****   ********  **  ** ******  ***")
    print("***   ***   **   **       **  **  ** **      ***")
    print("***   ***   *******       **  ****** ******  ***")
#soal 2
def npm2():
    npm=int(input("masukan NPM anda : "))
    TwoLastDigit=abs(npm)%100
    for i in range(TwoLastDigit):
       print("Halo, ", npm, " apa kabar ?")
#soal 3
def npm3():
    npm=int(input("Masukan NPM : "))
    key=str(npm%1000)
    print("Halo, "+str(npm)[4]+str(npm)[5]+str(npm)[6]+" apa kabar ?")

    for i in range(int(str(npm)[4])+int(str(npm)[5])+int(str(npm)[6])-1):
        print("Halo, "+str(npm)[4]+str(npm)[5]+str(npm)[6]+" apa kabar ?")
#soal 4
def npm4():
    npm=int(input("Masukan NPM : "))
    key=npm%1000
    str_key=str(key)
    print("Halo, "+str_key[0]+" apa kabar ?")
#soal 5
def npm5():
    i=0
    npm=input("Masukan NPM : ")
    while i<1:
        if len(npm) < 7:
            print("NPM Kurang dari 7 digit")
            npm=input("Masukan NPM : ")
        elif len(npm) > 7:
            print("NPM lebih dari 7 digit")
            npm=input("Masukan NPM : ")
        else:
            i=1
    a=npm[0]
    b=npm[1]
    c=npm[2]
    d=npm[3]
    e=npm[4]
    f=npm[5]
    g=npm[6]

    for x in a,b,c,d,e,f,g:
        print(x)

#soal 6
def npm6():
    i=0
    npm=input("Masukan NPM : ")
    while i<1:
        if len(npm) < 7:
            print("NPM Kurang dari 7 digit")
            npm=input("Masukan NPM : ")
        elif len(npm) > 7:
            print("NPM lebih dari 7 digit")
            npm=input("Masukan NPM : ")
        else:
            i=1
    a=npm[0]
    b=npm[1]
    c=npm[2]
    d=npm[3]
    e=npm[4]
    f=npm[5]
    g=npm[6]
    y=0

    for x in a,b,c,d,e,f,g:
        y+=int(x)
    print(y)
#soal 7
def npm7():
    i=0
    npm=input("Masukan NPM : ")
    while i<1:
        if len(npm) < 7:
            print("NPM Kurang dari 7 digit")
            npm=input("Masukan NPM : ")
        elif len(npm) > 7:
            print("NPM lebih dari 7 digit")
            npm=input("Masukan NPM : ")
        else:
            i=1
    a=npm[0]
    b=npm[1]
    c=npm[2]
    d=npm[3]
    e=npm[4]
    f=npm[5]
    g=npm[6]
    conv=1

    for x in a,b,c,d,e,f,g:
        conv*=int(x)
    print(conv)
#soal 8
def npm8():
    i=0
    npm=input("Masukan NPM : ")
    while i<1:
        if len(npm) < 7:
            print("NPM Kurang dari 7 digit")
            npm=input("Masukan NPM : ")
        elif len(npm) > 7:
            print("NPM lebih dari 7 digit")
            npm=input("Masukan NPM : ")
        else:
            i=1
    a=npm[0]
    b=npm[1]
    c=npm[2]
    d=npm[3]
    e=npm[4]
    f=npm[5]
    g=npm[6]
    conv=1

    for x in a,b,c,d,e,f,g:
    
        if int(x)%2==0:
            if int(x)==0:
                x=""
            print(x,end ="")
#soal 9
def npm9():
    i=0
    npm=input("Masukan NPM : ")
    while i<1:
        if len(npm) < 7:
            print("NPM Kurang dari 7 digit")
            npm=input("Masukan NPM : ")
        elif len(npm) > 7:
            print("NPM lebih dari 7 digit")
            npm=input("Masukan NPM : ")
        else:
            i=1
    a=npm[0]
    b=npm[1]
    c=npm[2]
    d=npm[3]
    e=npm[4]
    f=npm[5]
    g=npm[6]
    conv=1

    for x in a,b,c,d,e,f,g:
    
        if int(x)%2==1:
            print(x,end ="")
#soal 10
def npm10():
    i=0
    npm=input("Masukan NPM : ")
    while i<1:
        if len(npm) < 7:
            print("NPM Kurang dari 7 digit")
            npm=input("Masukan NPM : ")
        elif len(npm) > 7:
            print("NPM lebih dari 7 digit")
            npm=input("Masukan NPM : ")
        else:
            i=1
    a=npm[0]
    b=npm[1]
    c=npm[2]
    d=npm[3]
    e=npm[4]
    f=npm[5]
    g=npm[6]
    conv=1

    for x in a,b,c,d,e,f,g:    
        if int(x) > 1:
            for i in range(2,int(x)):
                if (int(x) % i) == 0:
                    break
            else:
                print(int(x),end =""),
          
	try :
		prima(npm)
		ganjil(npm)
		genap(npm)
		perkalian(npm)
		jumlah(npm)
		abc(npm)
		zero(npm)
		mulai(npm)
		lur(npm)
		npm()
	
	except ValueError :
		Print("parameter_tidak_di_isi")
\end{lstlisting}