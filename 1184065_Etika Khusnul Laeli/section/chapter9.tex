\chapter{Conditional Generative Adversarial Network}
Menggunakan sumber buku \cite{ahirwar2019gan}. Dengan source code yang ada di github awangga. 
https://github.com/awangga/Generative-Adversarial-Networks-Projects
Tujuan pembelajaran pada pertemuan pertama antara lain:
\begin{enumerate}
\item
Mengerti konsep Conditional Generative Adversarial Network
\item
Mengerti teknik CGAN
\item
Memahami penggunaan keras untuk cGAN
\end{enumerate}

Tugas dengan cara dikumpulkan dengan pull request ke github dengan menggunakan latex pada repo yang dibuat oleh asisten riset. Kode program menggunakan input listing ditaruh di folder src ekstensi .py dan dipanggil ke latex dengan input listings. \textbf{Tulisan dan kode tidak boleh plagiat}, menggunakan bahasa indonesia yang sesuai dengan gaya bahasa buku teks. Tidak menyertakan \textbf{pdf} kompilasi \textbf{diskon 50\%} nilainya.


\section{Soal Teori}
Teori diambil dari buku referensi \cite{ahirwar2019gan}  chapter 3.  Praktek teori penunjang yang dikerjakan total nilai 100 sebagai nilai terpisah dari praktek pada modul ini(nilai 9,09 per nomor kecuali nomor terakhir 30 sisanya merupakan penanganan error, untuk hari pertama) :
\begin{enumerate}
\item
Jelaskan dengan ilustrasi gambar sendiri apa perbedaan antara vanilla GAN dan cGAN.
\item
Jelaskan dengan ilustrasi gambar sendiri arsitektur dari Age-cGAN.
\item
Jelaskan dengan ilustrasi gambar sendiri arsitektur encoder network dari Age-cGAN.
\item
Jelaskan dengan ilustrasi gambar sendiri arsitektur generator network dari Age-cGAN.
\item
Jelaskan dengan ilustrasi gambar sendiri arsitektur discriminator network dari Age-cGAN.
\item
Jelaskan dengan ilustrasi gambar apa itu pretrained Inception-ResNet-2 Model.
\item
Jelaskan dengan ilustrasi gambar sendiri arsitektur Face recognition network Age-cGAN.
\item
Sebutkan dan jelaskan serta di sertai contoh-contoh tahapan dari Age-cGAN
\item
Berikan contoh perhitungan fungsi training objektif
\item
Berikan contoh dengan ilustrasi penjelasan dari Initial latent vector approximation
\item
Berikan contoh perhitungan latent vector optimization
\end{enumerate}


\section{Praktek Program}
Tugas nilai terpisah dari teori maksimal 100. Praktekkan dengan menjalankan kode program nya dan jelaskan (diperlihatkan di video youtube) dengan menggunakan bahasa yang mudah dimengerti dan bebas plagiat dan wajib diambil dari layar komputer sendiri masing masing nomor di bawah ini(nilai 15 masing masing pada hari kedua). Buka kode program pada repo Generative-Adversarial-Networks-Projects pada github awangga. Buka folder Chapter03. Kita praktekkan Age Conditional Generative Adversarial Networks (Age-cGAN). Dataset ada di https://drive.google.com/open?id=1NoV357ZvemE5dLCGySNTo6YNXsU8LyUs
\begin{enumerate}
\item
Jelaskan bagaimana cara ekstrak file dataset Age-cGAN menggunakan google colab
\item
Jelaskan bagaimana kode program bekerja untuk melakukan load terhadap dataset yang sudah di ekstrak, termasuk bagaimana penjelasan kode program perhitungan usia
\item
Jelaskan bagaimana kode program The Encoder Network bekerja dijelaskan dengan bahawa awam dengan ilustrasi sederhana
\item
Jelaskan bagaimana kode program The Generator Network bekerja dijelaskan dengan bahawa awam dengan ilustrasi sederhana
\item
Jelaskan bagaimana kode program The Discriminator Network bekerja dijelaskan dengan bahawa awam dengan ilustrasi sederhana
\item
Jelaskan bagaimana kode program Training cGAN bekerja dijelaskan dengan bahawa awam dengan ilustrasi sederhana
\item
Jelaskan bagaimana kode program Initial dan latent vector approximation bekerja dijelaskan dengan bahawa awam dengan ilustrasi sederhana
\end{enumerate}

\section{Penanganan Error}
Dari praktek pemrograman yang dilakukan di modul ini, error yang kita dapatkan(hasil komputer sendiri) di dokumentasikan dan di selesaikan(nilai 5 per error yang ditangani. Untuk hari kedua):

\begin{enumerate}
	\item skrinsut error
	\item Tuliskan kode eror dan jenis errornya
	\item Solusi pemecahan masalah error tersebut
\end{enumerate}