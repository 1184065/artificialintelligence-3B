\chapter{CNN}

Untuk pratikum saat ini menggunakan buku \textit{Python Artificial Intelligence Projects for Beginners}\cite{eckroth2018python}. Dengan praktek menggunakan python 3 dan editor spyder dan library python keras dan algoritma konvolusi.
Kode program ada di https://github.com/awangga/Python-Artificial-Intelligence-Projects-for-Beginners .
Tujuan pembelajaran pada pertemuan pertama antara lain:
\begin{enumerate}
\item
Mengerti konsep dasar Neural Network pada implementasi vektorisasi(Tokenize) di teks
\item
Mengerti teknik Konvolusi
\item
Memahami konsep deep learning
\end{enumerate}

Tugas dengan cara dikumpulkan dengan pull request ke github dengan menggunakan latex pada repo yang dibuat oleh asisten riset. Kode program menggunakan input listing ditaruh di folder src ekstensi .py dan dipanggil ke latex dengan input listings. \textbf{Tulisan dan kode tidak boleh plagiat}, menggunakan bahasa indonesia yang sesuai dengan gaya bahasa buku teks. Tidak menyertakan \textbf{pdf} kompilasi \textbf{diskon 50\%} nilainya.

\section{Teori}
Teori diambil dari buku referensi mengenai Neural Network dan deep learning dari dataset komentar di youtube dan HASYv2. Materi dan praktek bisa dilihat di youtube dosen.


\section{Soal Teori}
Praktek teori penunjang yang dikerjakan total nilai 100 sebagai nilai terpisah dari praktek pada modul ini(nilai 5 per nomor kecuali nomor terakhir 30 sisanya merupakan penanganan error, untuk hari pertama) :
\begin{enumerate}
\item
Jelaskan kenapa file teks harus di lakukan tokenizer. dilengkapi dengan ilustrasi atau gambar.
\item
Jelaskan konsep dasar K Fold Cross Validation pada dataset komentar Youtube pada kode listing \ref{lst:7.0}.dilengkapi dengan ilustrasi atau gambar.
\begin{lstlisting}[caption=K Fold Cross Validation,label={lst:7.0}]
kfold = StratifiedKFold(n_splits=5)
splits = kfold.split(d, d['CLASS'])
\end{lstlisting}

\item
Jelaskan apa maksudnya kode program \emph{for train, test in splits}.dilengkapi dengan ilustrasi atau gambar.
\item
Jelaskan apa maksudnya kode program \emph{train\_content = d['CONTENT'].iloc[train\_idx]} dan \emph{test\_content = d['CONTENT'].iloc[test\_idx]}. dilengkapi dengan ilustrasi atau gambar.
\item
Jelaskan apa maksud dari fungsi \emph{tokenizer = Tokenizer(num\_words=2000)} dan \emph{tokenizer.fit\_on\_texts(train\_content)}, dilengkapi dengan ilustrasi atau gambar.
\item
Jelaskan apa maksud dari fungsi \emph{d\_train\_inputs = tokenizer.texts\_to\_matrix(train\_content, mode='tfidf')} dan \emph{d\_test\_inputs = tokenizer.texts\_to\_matrix(test\_content, mode='tfidf')}, dilengkapi dengan ilustrasi kode dan atau gambar.
\item
Jelaskan apa maksud dari fungsi \emph{d\_train\_inputs = d\_train\_inputs/np.amax(np.absolute(d\_train\_inputs))} dan \emph{d\_test\_inputs = d\_test\_inputs/np.amax(np.absolute(d\_test\_inputs))}, dilengkapi dengan ilustrasi atau gambar.
\item
Jelaskan apa maksud fungsi dari \emph{d\_train\_outputs = np\_utils.to\_categorical(d['CLASS'].iloc[train\_idx])} dan \emph{d\_test\_outputs = np\_utils.to\_categorical(d['CLASS'].iloc[test\_idx])} dalam kode program, dilengkapi dengan ilustrasi atau gambar.
\item
Jelaskan apa maksud dari fungsi di listing \ref{lst:7.1}. Gambarkan ilustrasi Neural Network nya dari model kode tersebut.
\begin{lstlisting}[caption=Membuat model Neural Network,label={lst:7.1}]
       model = Sequential()
       model.add(Dense(512, input_shape=(2000,)))
       model.add(Activation('relu'))
       model.add(Dropout(0.5))
       model.add(Dense(2))
       model.add(Activation('softmax'))
\end{lstlisting}
\item
Jelaskan apa maksud dari fungsi di listing \ref{lst:7.2} dengan parameter tersebut.
\begin{lstlisting}[caption=Compile model,label={lst:7.2}]
	model.compile(loss='categorical_crossentropy', optimizer='adamax',
	                  metrics=['accuracy'])
\end{lstlisting}

\item
Jelaskan apa itu Deep Learning
\item
Jelaskan apa itu Deep Neural Network, dan apa bedanya dengan Deep Learning
\item
Jelaskan dengan ilustrasi gambar buatan sendiri(langkah per langkah) bagaimana perhitungan algoritma konvolusi dengan ukuran stride (NPM mod3+1) x (NPM mod3+1) yang terdapat max pooling.(nilai 30)

\end{enumerate}



\section{Praktek Program}
Tugas nilai terpisah dari teori maksimal 100. Praktekkan dan jelaskan dengan menggunakan bahasa yang mudah dimengerti dan bebas plagiat dan wajib skrinsut dari komputer sendiri masing masing nomor di bawah ini(nilai 5 masing masing pada hari kedua). Buka kode program pada repo Python-Artificial-Intelligence-Projects-for-Beginners pada github awangga. Buka folder Chapter04 file MathSymbols.py

\begin{enumerate}
\item Jelaskan kode program pada blok \# In[1]. Jelaskan arti dari setiap baris kode yang dibuat(harus beda dengan teman sekelas) dan hasil luarannya dari komputer sendiri.

\item Jelaskan kode program pada blok \# In[2]. Jelaskan arti dari setiap baris kode yang dibuat(harus beda dengan teman sekelas) dan hasil luarannya dari komputer sendiri.

\item Jelaskan kode program pada blok \# In[3]. Jelaskan arti dari setiap baris kode yang dibuat(harus beda dengan teman sekelas) dan hasil luarannya dari komputer sendiri.

\item Jelaskan kode program pada blok \# In[4]. Jelaskan arti dari setiap baris kode yang dibuat(harus beda dengan teman sekelas) dan hasil luarannya dari komputer sendiri.

\item Jelaskan kode program pada blok \# In[5]. Jelaskan arti dari setiap baris kode yang dibuat(harus beda dengan teman sekelas) dan hasil luarannya dari komputer sendiri.

\item Jelaskan kode program pada blok \# In[6]. Jelaskan arti dari setiap baris kode yang dibuat(harus beda dengan teman sekelas) dan hasil luarannya dari komputer sendiri.

\item Jelaskan kode program pada blok \# In[7]. Jelaskan arti dari setiap baris kode yang dibuat(harus beda dengan teman sekelas) dan hasil luarannya dari komputer sendiri.

\item Jelaskan kode program pada blok \# In[8]. Jelaskan arti dari setiap baris kode yang dibuat(harus beda dengan teman sekelas) dan hasil luarannya dari komputer sendiri.

\item Jelaskan kode program pada blok \# In[9]. Jelaskan arti dari setiap baris kode yang dibuat(harus beda dengan teman sekelas) dan hasil luarannya dari komputer sendiri.

\item Jelaskan kode program pada blok \# In[10]. Jelaskan arti dari setiap baris kode yang dibuat(harus beda dengan teman sekelas) dan hasil luarannya dari komputer sendiri.

\item Jelaskan kode program pada blok \# In[11]. Jelaskan arti dari setiap baris kode yang dibuat(harus beda dengan teman sekelas) dan hasil luarannya dari komputer sendiri.

\item Jelaskan kode program pada blok \# In[12]. Jelaskan arti dari setiap baris kode yang dibuat(harus beda dengan teman sekelas) dan hasil luarannya dari komputer sendiri.

\item Jelaskan kode program pada blok \# In[13]. Jelaskan arti dari setiap baris kode yang dibuat(harus beda dengan teman sekelas) dan hasil luarannya dari komputer sendiri.

\item Jelaskan kode program pada blok \# In[14]. Jelaskan arti dari setiap baris kode yang dibuat(harus beda dengan teman sekelas) dan hasil luarannya dari komputer sendiri.

\item Jelaskan kode program pada blok \# In[15]. Jelaskan arti dari setiap baris kode yang dibuat(harus beda dengan teman sekelas) dan hasil luarannya dari komputer sendiri.

\item Jelaskan kode program pada blok \# In[16]. Jelaskan arti dari setiap baris kode yang dibuat(harus beda dengan teman sekelas) dan hasil luarannya dari komputer sendiri.

\item Jelaskan kode program pada blok \# In[17]. Jelaskan arti dari setiap baris kode yang dibuat(harus beda dengan teman sekelas) dan hasil luarannya dari komputer sendiri.

\item Jelaskan kode program pada blok \# In[18]. Jelaskan arti dari setiap baris kode yang dibuat(harus beda dengan teman sekelas) dan hasil luarannya dari komputer sendiri.

\item Jelaskan kode program pada blok \# In[19]. Jelaskan arti dari setiap baris kode yang dibuat(harus beda dengan teman sekelas) dan hasil luarannya dari komputer sendiri.

\item Jelaskan kode program pada blok \# In[20]. Jelaskan arti dari setiap baris kode yang dibuat(harus beda dengan teman sekelas) dan hasil luarannya dari komputer sendiri.

\end{enumerate}


\section{Penanganan Error}
Dari praktek pemrograman yang dilakukan di modul ini, error yang kita dapatkan(hasil komputer sendiri) di dokumentasikan dan di selesaikan(nilai 5 per error yang ditangani. Untuk hari kedua):

\begin{enumerate}
	\item skrinsut error
	\item Tuliskan kode eror dan jenis errornya
	\item Solusi pemecahan masalah error tersebut
\end{enumerate}

\section{Presentasi Tugas}
Pada pertemuan ini, diadakan tiga penilaiain yaitu penilaian untuk tugas mingguan hari pertama dan hari kedua yang terpisah masing-masing dengan nilai maksimal 100. Kemudian dalam satu minggu kedepan maksimal sebelum waktu mata kuliah kecerdasan buatan. Ada presentasi kematerian dengan nilai presentasi yang terpisah dengan nilai maksimal 100. Jadi ada tiga komponen penilaiain pada pertemuan ini yaitu :
\begin{enumerate}
	\item tugas teori hari pertama(maks 100) modul ini
	\item tugas praktek hari kedua modul ini(maks 100)
	\item Presentasi tugas penjelasan CNN dan deep learning, Mempraktekkan kode python dan menjelaskan cara kerjanya(maks 100).
\end{enumerate}
Waktu presentasi pada jam kerja di IRC. Kriteria penilaian presentasi sangat sederhana, presenter akan ditanyai 20(10 praktek dan 10 teori) pertanyaan tentang pemahamannya menggunakan python untuk kecerdasan buatan dan teori konvolusi dan deep learning. jika presenter tidak bisa menjawab satu pertanyaan asisten maka nilai nol. Jika semua pertanyaan bisa dijawab maka nilai 100. Presentasi bisa diulang apabila gagal, sampai bisa mendapatkan nilai 100 dalam waktu satu minggu kedepan.


