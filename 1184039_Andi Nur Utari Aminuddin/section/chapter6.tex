\chapter{MFCC dan Neural Network}

Untuk pratikum saat ini menggunakan buku \textit{Python Artificial Intelligence Projects for Beginners}\cite{eckroth2018python}. Dengan praktek menggunakan python 3 dan editor spyder dan library python keras dan librosa.
Kode program ada di https://github.com/awangga/Python-Artificial-Intelligence-Projects-for-Beginners .
Tujuan pembelajaran pada pertemuan pertama antara lain:
\begin{enumerate}
\item
Mengerti konsep dasar MFCC untuk vektorisasi suara
\item
Mengerti teknik Neural Network dari hasil MFCC
\item
Memahami konsep pembobotan dan fungsi aktifasi
\end{enumerate}

Tugas dengan cara dikumpulkan dengan pull request ke github dengan menggunakan latex pada repo yang dibuat oleh asisten riset. Kode program menggunakan input listing ditaruh di folder src ekstensi .py dan dipanggil ke latex dengan input listings. \textbf{Tulisan dan kode tidak boleh plagiat}, menggunakan bahasa indonesia yang sesuai dengan gaya bahasa buku teks. Tidak menyertakan \textbf{pdf} kompilasi \textbf{diskon 50\%} nilainya.

\section{Teori}
Teori diambil dari buku referensi mengenai Neural Network dan MFCC dari dataset suara. Materi dan praktek bisa dilihat di youtube dosen.


\section{Soal Teori}
Praktek teori penunjang yang dikerjakan(nilai 5 per nomor, untuk hari pertama) :
\begin{enumerate}
\item
Jelaskan kenapa file suara harus di lakukan MFCC. dilengkapi dengan ilustrasi atau gambar.
\item
Jelaskan konsep dasar neural network.dilengkapi dengan ilustrasi atau gambar.
\item
Jelaskan konsep pembobotan dalam neural network.dilengkapi dengan ilustrasi atau gambar.
\item
Jelaskan konsep fungsi aktifasi dalam neural network. dilengkapi dengan ilustrasi atau gambar.
\item
Jelaskan cara membaca hasil plot dari MFCC,dilengkapi dengan ilustrasi atau gambar.
\item
Jelaskan apa itu one-hot encoding,dilengkapi dengan ilustrasi kode dan atau gambar.
\item
Jelaskan apa fungsi dari np.unique dan to\_categorical dalam kode program,dilengkapi dengan ilustrasi atau gambar.
\item
Jelaskan apa fungsi dari Sequential dalam kode program,dilengkapi dengan ilustrasi atau gambar.
\end{enumerate}



\section{Praktek Program}
Tugas anda adalah,praktekkan dan jelaskan dengan menggunakan bahasa yang mudah dimengerti dan bebas plagiat dan wajib skrinsut dari komputer sendiri masing masing nomor di bawah ini(nilai 5 masing masing pada hari kedua).

\begin{enumerate}
\item Jelaskan isi dari data GTZAN Genre Collection dan data dari freesound. Buat kode program untuk meload data tersebut untuk digunakan pada MFCC. Jelaskan arti dari setiap baris kode yang dibuat(harus beda dengan teman sekelas).

\item Jelaskan perbaris kode program dengan kata-kata dan dilengkapi ilustrasi gambar fungsi dari display\_mfcc() .

\item Jelaskan perbaris kode program dengan kata-kata dan dilengkapi ilustrasi gambar fungsi dari extract\_features\_song(). Jelaskan juga mengapa yang diambil 25.000 baris pertama?

\item Jelaskan perbaris kode program dengan kata-kata dan dilengkapi ilustrasi gambar fungsi dari generate\_features\_and\_labels().

\item Jelaskan dengan kata dan praktek kenapa penggunaan fungsi generate\_features\_and\_labels() sangat lama ketika meload dataset genre. Tunjukkan keluarannya dari komputer sendiri dan artikan maksud setiap luaran yang didapatkan.

\item Jelaskan kenapa harus dilakukan pemisahan data training dan data set sebesar 80 persen? Praktekkan dengan kode dan Tunjukkan keluarannya dari komputer sendiri dan artikan maksud setiap luaran yang didapatkan.

\item Praktekkan dan jelaskan masing-masing parameter dari fungsi Sequential().Tunjukkan keluarannya dari komputer sendiri dan artikan maksud setiap luaran yang didapatkan.

\item Praktekkan dan jelaskan masing-masing parameter dari fungsi compile().Tunjukkan keluarannya dengan fungsi summary dari komputer sendiri dan artikan maksud setiap luaran yang didapatkan.

\item Praktekkan dan jelaskan masing-masing parameter dari fungsi fit().Tunjukkan keluarannya dari komputer sendiri dan artikan maksud setiap luaran yang didapatkan.

\item Praktekkan dan jelaskan masing-masing parameter dari fungsi evaluate().Tunjukkan keluarannya dari komputer sendiri dan artikan maksud setiap luaran yang didapatkan.

\item Praktekkan dan jelaskan masing-masing parameter dari fungsi predict().Tunjukkan keluarannya dari komputer sendiri dan artikan maksud setiap luaran yang didapatkan.

\end{enumerate}


\section{Penanganan Error}
Dari praktek pemrograman yang dilakukan di modul ini, error yang kita dapatkan(hasil komputer sendiri) di dokumentasikan dan di selesaikan(nilai 5 per error yang ditangani. Untuk hari kedua):

\begin{enumerate}
	\item skrinsut error
	\item Tuliskan kode eror dan jenis errornya
	\item Solusi pemecahan masalah error tersebut
\end{enumerate}

\section{Presentasi Tugas}
Pada pertemuan ini, diadakan dua penilaiain yaitu penilaian untuk tugas mingguan seperti sebelumnya dengan nilai maksimal 100. Kemudian dalam satu minggu kedepan maksimal sebelum waktu mata kuliah kecerdasan buatan. Ada presentasi kematerian dengan nilai presentasi yang terpisah masing-masing 100. Jadi ada dua komponen penilaiain pada pertemuan ini yaitu :
\begin{enumerate}
	\item tugas minggu hari ini dan besok (maks 100). pada chapter ini
	\item presentasi tugas kode MFCC dan Neural Network (maks 100). Mempraktekkan kode python dan menjelaskan cara kerjanya.
\end{enumerate}
Waktu presentasi pada jam kerja di IRC. Kriteria penilaian presentasi sangat sederhana, presenter akan ditanyai 20(10 praktek dan 10 teori) pertanyaan tentang pemahamannya menggunakan python untuk kecerdasan buatan. jika presenter tidak bisa menjawab satu pertanyaan asisten maka nilai nol. Jika semua pertanyaan bisa dijawab maka nilai 100. Presentasi bisa diulang apabila gagal, sampai bisa mendapatkan nilai 100 dalam waktu satu minggu kedepan.


